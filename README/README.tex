\documentclass{article}
\usepackage[utf8]{inputenc}

\title{Automated Reasoning}
\author{Brian Kuhns}
\date{April 2019}

\begin{document}

\maketitle

\section{Introduction}

  Someone better at writing do this part

\section{Algorithm}

The algorithm is an extension of ordered resolution,
which attempts to find finite representations of infinite clause families,
and proceed with ordered resolution on these families.

\subsection{Anatomy of a Scheme}

A scheme contains two parts, a list of implications and a base.
The list of implications is stored as a list of pairs of clauses,
signifying that the left clause implies the right clause, hereafter the hypothesis and the conclusion.
The base is a clause.

An implication is syntactically valid if the hypothesis subsumes the conclusion.
A scheme is syntactically valid if each implication is valid and
the hypothesis of each implication subsumes the base.

A scheme is said to contain a clause iff it can be derived by applying some finite sequence of the 
implications to the base clause.

\subsection{Learning Implications}

Learning candidates is done in two phases: finding candidates and constructing implications from them.
 
\subsubsection{Candidates}

A candidate is a start clause, a chain of clauses, and an end clause:
where the start clause can be, resolved with the clauses in the chain in sequence to produce the end clause.

The important property to have is that
if the clause can be resolved through the chain infinitely a candidate will be examined.
To ensure the first any time a chain appears twice in the history of a clause
consider the clause before the first appearance of the chain and the chain as a candidate.
In order to learn simple implications more quickly you can also form two candidates after each resolution,
consisting of each as a base and the other as a single clause chain.

\subsubsection{Implications}

When given a candidate the important property to ensure is that if it can be resolved indefinitely,
then a scheme which subsumes all of these clauses is derived.

In order to ensure this, take the start clause and create all of its resolutions with the chain.
Find a complete set of mutually Subsumptive clauses.
Resolve that clause with the clauses in the chain.
For each clause you get through that resolution create a new implication from the MGMSC to the resolved clause.

This implication must be valid as by assuming the hypothesis the conclusion was derived within the problem.
This implication will be put in a scheme with the start clause as the base because its hypothesis (the MGMSC) subsumes the start clause.
This implication will subsume the chain of resolutions because when the implication is applied to the base,
  it will produce the end clause.

The complete set of mutually subsumptive clauses of two clauses must have the following properties
\begin{itemize}
  \item Each element must subsume both clauses
  \item Some element must subsume any clause which subsumes both clauses
\end{itemize}

In order to find the  find



\end{document}
